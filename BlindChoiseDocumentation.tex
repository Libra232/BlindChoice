\documentclass{article}
\usepackage{hyperref}
\usepackage{listings}
\usepackage{tcolorbox}


\title{Blind Choice Documentation}

\author{Aiupova Lisa,
Milova Irina,
Titov Maxim}

\date{2024}


\begin{document}

\maketitle

\tableofcontents

\section{Project Overview}

\textbf{BlindChoice}
is a C++ application that manages book and film choices with a GUI interface. It provides functionalities for editing book and film details and saving choices.



\section{Getting Started}


\subsection{Prerequisites}

\begin{itemize}
    \item \textbf{C++ Compiler}:
    Ensure you have a C++ compiler installed.
    \item \textbf{Qt Framework}:
    The project uses Qt for the GUI components.
    \item \textbf{CMake}:
    Used for building the project.
\end{itemize}


\subsection{Installation}

\begin{enumerate}
    \item Clone the repository:
    git clone https://github.com/Libra232/BlindChoice.git
    cd BlindChoice
    \item Create a build directory and navigate into it:
    mkdir build
    cd build
    \item Configure the project using CMake:
    cmake ..
    \item Build the project:
    make
\end{enumerate}


\subsection{Quick Start Guide}

\begin{enumerate}
    \item Run the executable:
    ./BlindChoice
    \item The application window will open, allowing you to manage book/film choices.
\end{enumerate}




\section{Usage}


\subsection{Command-Line Arguments}
Currently, there are no specific command-line arguments required to run the application.


\subsection{Configuration}
Configuration is handled through the GUI. No external configuration files are necessary.


\subsection{Examples}

\begin{itemize}
    \item \textbf{Adding a Book/Film}:
    \begin{enumerate}
        \item Open the application.
        \item Click on "Add Book/Film" and fill in the details.
    \end{enumerate}
    \item \textbf{Editing a Book/Film}:
    \begin{enumerate}
        \item Select a book from the list.
        \item Click on "Edit Book/Film" and modify the details.
    \end{enumerate}
\end{itemize}



\section{Code Documentation}


\subsection{Code Structure}

\begin{itemize}
    \item {main.cpp}:
    Entry point of the application.
    \item {mainwindow.cpp/h}:
    Contains the main window logic and definitions.
    \item {bookeditdialog.cpp/h}:
    Manages the book editing dialog functionalities.
    \item {storage.cpp/h}:
    Handles data storage and retrieval.
    \item {blchoice.h}:
    Main application class definition.
\end{itemize}


\subsection{Modules and Classes}


\begin{itemize}
    \item \textbf{MainWindow}:
    Main application window class.
    \item \textbf{BookEditDialog}:
    Dialog for adding/editing book details.
    \item \textbf{Storage}:
    Manages saving and loading data.
\end{itemize}


\subsection{Inline Comments}

Important functions and complex code blocks are commented within the source files.



\section{Contributing}


\subsection{Guidelines}

\begin{enumerate}
    \item Fork the repository.
    \item Create a new branch:
    git checkout -b feature-name
    \item Commit your changes:
    git commit -m "Add feature"
    \item Push to the branch:
    git push origin feature-name
    \item Create a pull request.
\end{enumerate}


\subsection{Bug Reporting}

Report bugs by opening an issue on the \href{https://github.com/Libra232/BlindChoice/issues}{GitHub Issues} page.


\subsection{Development Workflow}

\begin{itemize}
    \item Ensure all changes are tested before submitting a pull request.
    \item Follow the code style and naming conventions used in the project.
\end{itemize}



\section{Testing}


\subsection{Test Suite}

Currently, no automated test suite is provided.


\subsection{Manual Testing}

\begin{enumerate}
    \item Compile the project.
    \item Run the application and manually verify the functionalities.
\end{enumerate}



\section{Licensing}


\textbf{BlindChoice} is distributed under the MIT License.



\section{Additional Resources}


\subsection{FAQ}

\begin{itemize}
    \item \textbf{Q: How do I add a new book/film?}
    \begin{itemize}
        \item \textbf{A:} Use the "Add Book/Film" button in the main window.
    \end{itemize}
\end{itemize}


\subsection{References}

\begin{itemize}
    \item \href{https://doc.qt.io/}{Qt Documentation}
\end{itemize}


\subsection{Contact Information}

For further assistance, contact the project maintainers through the \href{https://github.com/Libra232/BlindChoice/issues}{GitHub Issues} page.



\section{Appendix}


\subsection{Glossary}

\begin{itemize}
    \item \textbf{GUI}: Graphical User Interface.
    \item \textbf{CMake}: Cross-platform make tool.
\end{itemize}


\subsection{Changelog}

Refer to the commit history for a detailed list of changes.

\end{document}

